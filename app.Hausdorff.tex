%!TEX root = main.tex
\chapter{The Hausdorff Distance for Polytopic Sets}\label{appendix:hausdorff:distance:stuff}
\resetcounters
%
In the previous section we have discussed the Minkowski addition and the Pontryagin subtraction, at various instances throughout this thesis we have presented statements that involved set sequences that attained their limit in a finite number of iterations.
%
In Lemma~\ref{thm:mrpi:main:theorem},~\ref{thm:finite:determinatbility:MRPI:state:dependent} and~\ref{thm:scaled:distrubances:mrpi} we had set sequences which converged in a finite number of iterations, however the bounds we presented in the respective proofs where largely abstracted, we used $P$-balls to connect the dynamic behaviour of the underlying system with the set sequence.
%
Here we present the tools necessary to analyse the presented convergence results in the context of metric spaces, where convergence has a precise meaning.
%
We will see why we had to use the previously presented approach to obtain sensible bounds.
%
\\[1em]
%
\mysplit Let~$\mathcal X,\mathcal Y$ be elements of the metric space $(\mathcal K,d)$, i.e. $\mathcal X\subset \mathcal K$ and $\mathcal Y\subset \mathcal K$ and the set $\mathcal K$ is equipped with its metric $d(\cdot,\cdot)$.
%
Then the Hausdorff distance between~$\X$ and $\Y$ is given by
%
\begin{equation}
	d(\mathcal X,\mathcal Y) = \sup\left\{\sup_{x\in\mathcal X}\inf_{y\in\mathcal Y} d(x,y), \sup_{y\in\mathcal Y}\inf_{x\in\mathcal X} d(x,y) \right\}.
\end{equation}
%
The Hausdorff distance is a metric on the space of compact, non-empty sets.
%
We illustrate the basic idea of the Hausdorff distance in Figure~\ref{fig:appendix:hausdorff:illustration}
%
\begin{figure}\centering
\begin{tikzpicture}
% \draw (0,0) grid (10,10);
% \foreach \x in {0,1,...,10} \draw (\x,0) node[below] {$\x$};
% \foreach \y in {1,2,...,10} \draw (0,\y) node[left] {$\y$};
\draw (5,5) node {\includegraphics{Pix/Hausdorff.pdf}};
\draw[latex'-latex'] (1.5,1.87) -- node[sloped,above] {$d(\textcolor{blue}{\X},\textcolor{red}{\Y})$} (3.2,4.32) node {};
\draw[latex'-latex'] (3.2,4.32) -- node[sloped,below] {$d(\textcolor{blue}{x^\ast},\textcolor{red}{y^\ast})$} (1.5,1.87) node {};
\draw[red] (3,7.5) node {$\Y$};
\draw[blue] (5,5) node {$\X$};
\draw[red] (1.5,1.87) circle[radius=1.5pt] node[below] {$y^\ast$};
\draw[blue] (3.2,4.32) circle[radius=1.5pt] node[above] {$x^\ast$};
\end{tikzpicture}
\caption[Illustration of the Hausdorff metric]{The Hausdorff distance measures the greatest distance between a point ~$x\in\X$ and the set~$\Y$ and vice versa. 
%
Naturally, if $\X$ and $\Y$ are closed, there exist two points $x^\ast\in\X$ and $y^\ast\in\Y$ such that $d(\X,\Y)=d(x^\ast,y^\ast)$, characterising these points will turn out central in the proceeding.}
\label{fig:appendix:hausdorff:illustration}
\end{figure}
%
Alternatively, the Hausdorff distance can be defined using the dilatation with a metric ball $\ball(\rho)=\{x:d(0,x)\leq\rho\}$, we denote $\X_\rho = \X\oplus\ball(\rho)$ and we have
%
\begin{equation}\label{eq:Hausdorff:bound:using:balls}
	d(\mathcal X,\mathcal Y) = \left\{\begin{aligned}\inf&\quad\rho\\\text{s.t.}&\quad \X\subseteq\Y_\rho\\&\quad\Y\subseteq\X_\rho\end{aligned}\right.
\end{equation}
%
With this definition it is less obvious to see that for $\X,\Y$ compact there exist to points~$x^\ast,y^\ast$ in the respective sets at which the the distance is attained~$d(x^\ast,y^\ast)=d(\X,\Y)$, the space matric~$d:\mathcal K\times\mathcal K\rightarrow [0,\infty)$ is hidden inside the introduced ball, in particular the space~$\mathcal K$ has to have a zero element.
%
However, we do not deal with general metric spaces in this thesis, so we can assume that the ball is the Euclidean norm ball $\ball_2(\cdot)$.
%
\\[1em]
%
In order to handle any of the proposed set iterations~\eqref{eq:definition:Ek:sequence:for:set:iteration},~\eqref{eq:definition:Rk:dual:of:Ek} or~\eqref{eq:definition:Dk:sequence:scaled:disturbances} we need to be able to characterise the intersection of sets:
%
\begin{thm}\label{thm:non:empty:intersection}
Let $\mathcal X$ and $\mathcal Y$ be such that their intersection has a non-empty interior, then there exits a positive number $\delta>0$ such that all $\mathcal Z$ with $d(\mathcal Z,\mathcal X)\leq\delta$ have a non-empty the intersection with $\mathcal Y$, i.e. $\mathcal Z\cap\mathcal Y\neq\emptyset$.
\end{thm}
%
\begin{proof}
Let $p\in\mathcal X\cap\mathcal Y$ and let $\delta$ denote the radius of the largest ball centred at $p$ contained in $\mathcal X\cap\mathcal Y$.
%
Since $d(\mathcal X,\mathcal Z)\leq\delta$ there exists a point $\tilde p\in\mathcal Z$ such that $d(p,\tilde p)\leq \delta$, hence $\tilde p$ is contained in the $\delta$-ball around $p$, which is contained in $\mathcal X\cap\mathcal Y$ and therefore $\tilde p\in\mathcal Z\cap\mathcal Y\neq\emptyset$.
\end{proof}
%
\noindent Furthermore we can show:
%
\begin{thm}
Let $\mathcal X,\mathcal Y$ and $\mathcal Z$ as in Lemma~\ref{thm:non:empty:intersection}, then we have $d(\mathcal X\cap\mathcal Y,\mathcal Z\cap\mathcal Y)\leq d(\mathcal X,\mathcal Z)$.
\end{thm}
%
\begin{proof}
Let $x_k\in\mathcal X,z_k\in\mathcal Z$ be sequences such that $d(x_k,z_k)\to d(\mathcal X,\mathcal Z)$.
%
Either $x_k$ and $z_k$ can be chosen in the intersection $x_k,z_k\in\mathcal X\cap\mathcal Z$ in which case the same sequences yield $d(x_k,z_k)\to d(\mathcal X\cap\mathcal Y,\mathcal Z\cap\mathcal Y) = d(\mathcal X,\mathcal Z)$, or at least one of the sequences has to be chosen in the complement.
%
This means the defining elements of the distance between $\mathcal X$ and $\mathcal Z$ do not lie in $\mathcal Y$ therefore the distance of $\mathcal X\cap\mathcal Y$ and $\mathcal Z\cap\mathcal Y$ is smaller, i.e. $d(\mathcal X\cap\mathcal Y,\mathcal Z\cap\mathcal Y)< d(\mathcal X,\mathcal Z)$.
\end{proof}
%
\noindent The main result for intersections of sets is the following statement:
%
\begin{thm}
Let $\mathcal X,\tilde{\mathcal X},\mathcal Y$ and $\tilde{\mathcal Y}$ be compact, non-empty such that $d(\mathcal X,\tilde{\mathcal X})\leq\delta_1$ and $d(\mathcal Y,\tilde{\mathcal Y})\leq\delta_2$, then the intersection $\mathcal X\cap \mathcal Y$ has a distance to $\tilde{\mathcal X}\cap\tilde{\mathcal Y}$ no larger than $\delta_1+\delta_2$, i.e. $d(\mathcal X\cap \mathcal Y,\tilde{\mathcal X}\cap\tilde{\mathcal Y})\leq\delta_1+\delta_2$.
\end{thm}
%
\begin{proof}
We exploit the fact that the Hausdorff distance is a metric on the set of compact, non-empty spaces, in particular it satisfies the triangle inequality $d(A,C)\leq d(A,B)+d(B,C)$.
%
By applying the previous lemma twice we have 
%
\begin{equation}
	d(\mathcal X\cap \mathcal Y,\tilde{\mathcal X}\cap\tilde{\mathcal Y}) \leq d(\mathcal X\cap \mathcal Y,\mathcal X\cap\tilde{\mathcal Y}) + d(\mathcal X\cap\tilde{\mathcal Y},\tilde{\mathcal X}\cap\tilde{\mathcal Y}) \leq d(\mathcal Y,\tilde{\mathcal Y}) + d(\mathcal X,\tilde{\mathcal X}) \leq \delta_2 + \delta_1.
\end{equation}
\end{proof}
%
\noindent For a more complete presentation of properties of the Hausdorff metric we cite three statements from~\cite{Hadwiger:1957}:
%
\begin{thm}
Let~$\X,\Y, \Z$ and $\mathcal R$ denote elements of the metric space~$\mathcal K$, then the following bounds hold:
%
\begin{align}
	d(\X\cup\Y,\Z\cup\mathcal R)&\leq\max\{d(\X,\Z),d(\Y,\mathcal R)\}\\
	d(\X\oplus\Y,\Z\oplus\mathcal R)&\leq d(\X,\Z) + d(\Y,\mathcal R)\\
	d(\X_\rho,\Y_\rho)&\leq d(\X,\Y)
\end{align}
\end{thm}
%
\begin{proof}
Let~$\alpha=d(\X,\Z),\beta=d(\Y,\mathcal R)$ and $\gamma=\max\{\alpha,\beta\}$, by the alternative definition of the Hausdorff distance we have $\X\subseteq\Z_\alpha$ and $\Y\subseteq\mathcal R_\beta$ and hence $\X\cup\Y\subseteq\Z_\alpha\cup\mathcal R_\beta\subseteq(\Z\cup\mathcal R)_\gamma$. 
%
With an analogous argument we obtain $\Z\cup\mathcal R\subseteq(\X\cup\Y)_\gamma$ and hence the first identity.
%
For the second statement we fist show that $\X_\alpha\oplus\Y_\beta\subseteq(\X\oplus\Y)_{\alpha+\beta}$:
%
The set~$\X_\alpha\oplus\Y_\beta = (\X\oplus\Y)\oplus(\ball(\alpha)\oplus\ball(\beta))$, but since the metric satisfies the triangle inequality~$d(x+y,0)\leq d(x,0)+d(y,0)$ we have that $\ball(\alpha)\oplus\ball(\beta)\subseteq\ball(\alpha+\beta)$.
%
With this we first get~$\X\oplus\Y\subseteq\Z_\alpha\oplus\mathcal R_\beta\subseteq(\Z\oplus\mathcal R)_{\alpha+\beta}$, and analogously~$\Z\oplus\mathcal R\subseteq(\X\oplus\Y)_{\alpha+\beta}$ which proves the statement.
%
The third statement is a slight extension of the second case since $\X_\rho=\X\oplus\ball(\rho)$ and $\Y_\rho = \Y\oplus\ball(\rho)$ hence $d(\X_\rho,\Y_\rho)\leq d(\X,\Y)+d(\ball(\rho),\ball(\rho))=d(\X,\Y)$.
\end{proof}
%
\noindent Notice that equality does not hold for the third statement in general, for this consider $\X=\Y$ except for one of the sets to have a hole, for $\rho$ large enough the hole is closed and the distance between~$\X_\rho$ and $\Y_\rho$ vanishes, in particular this example illustrates that without convexity of both sets~$\X,\Y$ the map $\rho\rightarrow d(\X_\rho,\Y_\rho)$ is not continuous.
%
\\[1em]
%
\noindent\mysplit Here we will only deal with convex sets $\mathcal X,\mathcal Y$, for convex sets we know that the distance $d(x,\mathcal Y) = \inf_{y\in\mathcal Y}d(x,y)$ is convex, i.e. $d(\lambda a + (1-\lambda)b,\mathcal Y)\leq\lambda d(a,\mathcal Y)+(1-\lambda) d(b,\mathcal Y)$.
%
Recall that for convex sets we have the set of extremal points $\ext(\mathcal X)$ which contains all points that can not be expressed as the convex combination of other elements in the set, i.e.
%
\begin{equation}
	\ext(\mathcal X) = \{x\in\mathcal X: \not\exists a,b\in\mathcal X\setminus\{x\} ,\lambda\in(0,1)\quad x=\lambda a + (1-\lambda) b \}
\end{equation}
%
One obvious fact about the set of extremal points with respect to the distance function can be summarised by:
%
\begin{thm}
For a convex set $\mathcal Y\subset X$ the distance function $d(x,\mathcal Y)$ has the upper bound
%
\begin{equation}
	d(x,\mathcal Y)\leq d(x,\ext(\mathcal Y)).
\end{equation}
\end{thm}
%
\begin{proof}
%
The trivial relationship
%
\begin{equation}
	d(x,\mathcal Y) = \inf_{y\in\mathcal Y} d(x,y) \leq \inf_{y\in\ext(\mathcal Y)} d(x,y) = d(x,\ext(\mathcal Y)).
\end{equation}
%
follows from $\ext(\mathcal Y)\subseteq \mathcal Y$.
\end{proof}
%
\noindent Putting this all together we can prove the following statement:
%
\begin{thm}\label{thm:Hausdorff:convex:sets:extremal:points}
Let $\mathcal X,\mathcal Y\subset X$ be convex sets, then $d(\mathcal X,\mathcal Y)\leq d(\ext(\mathcal X),\ext(\mathcal Y))$.
%
Furthermore $d(\mathcal X,\mathcal Y) = \sup\left\{d(\mathcal X,\text{ext}(\mathcal Y)),d(\text{ext}(\mathcal X),\mathcal Y)\right\}$.
\end{thm}
%
\begin{proof}
This now just follows the definition of the Hausdorff distance
%
\begin{multline}
	d(\mathcal X,\mathcal Y) = \sup\left\{\sup_{x\in\mathcal X} \underbrace{d(x,\mathcal Y)}_{\leq d(x,\ext(\mathcal Y))}, \sup_{y\in\mathcal Y} \underbrace{d(\mathcal X,y)}_{\leq d(\ext(\mathcal X),y)} \right\}\\ \leq 
	\sup\left\{\sup_{x\in\mathcal X} d(x,\ext(\mathcal Y)), \sup_{y\in\mathcal Y}  d(\ext(\mathcal X),y) \right\} = d(\ext(\mathcal X),\ext(\mathcal Y)).
\end{multline}
%
The second statement follows directly from the convexity of the sets: The optimisers are attained on the boundary~$\partial \mathcal X\supseteq\text{ext}(\mathcal X)$ and~$\partial\mathcal Y\supseteq\text{ext}(\mathcal Y)$.
%
Any connected subset~$U \subset \partial\mathcal Y\setminus\text{ext}(\mathcal Y)$ can be expressed as a convex combination of extremal points, i.e. it is a hyperplane. 
%
If both optimisers $x,y$ with $d(x,y) = d(\mathcal X,\mathcal Y)$ lie on (parallel) hyperplanes, i.e. both points are boundary points but not extremal, then continuing along the hyperplane does not change the distance.
%
Therefore choosing any extremal point on the boundary of the hyperplane with the same distance yields the desired result.
%
\end{proof}
%
\noindent This relationship becomes computationally convenient when the sets of extremal points are collections of points, i.e. when $\mathcal X$ and $\mathcal Y$ are polytopic. 
%
In this case we have the relationship
%
\begin{thm}\label{thm:Hausdorff:for:polytopes}
Let $\mathcal X = \conv\{v_i\}$ and $\mathcal Y = \conv\{w_i\}$ then
%
\begin{equation}\label{eq:maximal:distance:between:vertices}
	d(\mathcal X,\mathcal Y) \leq \max\left\{\max_i\min_j d(v_i,w_j),\max_j\min_i d(v_i,w_j)\right\}.
\end{equation}
\end{thm}
%
\begin{proof}
All sets are now finite dimensional and closed and bounded, therefore all optima are attained.
%
The rest follows from Lemma~\ref{thm:Hausdorff:convex:sets:extremal:points}.
\end{proof}
%
\noindent The result is illustrated in Figure~\ref{fig:appendix:hausdorff:for:polytopes}.
%
\\[2em]
%
\begin{figure}\centering
\begin{tikzpicture}
\draw[latex'-latex'] (-1,1) -- (-.5,.5);
\draw[latex'-latex'] (-1,1) -- (-1,0);
\draw[latex'-] (-.7071,.7071) to [out=45,in=180] (2,2) node[right] {$\gamma$};
\draw[latex'-] (-1,.5) to [out=150,in=0] (-2,1) node[left] {$\rho$};
\draw[latex'-] (1.2,2-1.2) to [out=30,in=180] (2,1) node[right] {\textcolor{green!80!black}{$\X_\gamma$}};
\draw[latex'-] (1.2,-2.4142+1.2) to [out=-30,in=180] (2,-1.5) node[right] {\textcolor{magenta!70!black}{$\X_\rho$}};
\draw[latex'-] (0,0) to [out=-15,in=180] (2,-.5) node[right] {\textcolor{blue}{$\X$}};
\draw[latex'-] (-.7071,-.7071) to [out=210,in=0] (-2,-1) node[left] {\textcolor{red}{$\Y$}};
\draw[blue] (  1.0000,   0.0000) -- (  0.0000,   1.0000) -- ( -1.0000,   0.0000) -- (  0.0000,  -1.0000) -- (  1.0000,   0.0000) -- cycle;
\draw[rounded corners=0.7071cm,green!80!black] (  2.0000,   0.0000) -- (  0.0000,   2.0000) -- ( -2.0000,   0.0000) -- (  0.0000,  -2.0000)  -- cycle;
\draw[red] ( -1.0000,  -1.0000) -- (  1.0000,  -1.0000) -- (  1.0000,   1.0000) -- ( -1.0000,   1.0000) -- ( -1.0000,  -1.0000) -- cycle;
\draw[rounded corners=1cm,magenta!70!black] (  2.4142,   0.0000) -- (  0.0000,   2.4142) -- ( -2.4142,   0.0000) -- (  0.0000,  -2.4142)-- cycle;
\end{tikzpicture}
\caption[Hausdorff distance for polytopes.]{The two polytopes~\textcolor{blue}{$\X$} and~\textcolor{red}{$\Y$} are shown together with the vector~$y^\ast-x^\ast$ defining the Hausdorff distance~$\gamma=d(\textcolor{blue}{\X},\textcolor{red}{\Y})$, furthermore the vector~$v_{i^\ast}-w_{j^\ast}$ where $\rho=\max\left\{\max_i\min_j d(v_i,w_j),\max_j\min_i d(v_i,w_j)\right\}$ is attained.
%
The sets~$\X_\gamma$ and $\X\rho$ are shown to illustrate the statement of Lemma~\ref{thm:Hausdorff:for:polytopes}, i.e.~$\Y\subseteq\X_\gamma\subseteq\X_\rho$.}
\label{fig:appendix:hausdorff:for:polytopes}
\end{figure}
%
\noindent\mysplit We can now make try to use these results on the set iterations involved in the computation of the maximal robust positive invariant set~$\X^\infty_{\max}$ in~\eqref{eq:definition:Ek:sequence:for:set:iteration}, notice that~\eqref{eq:definition:Ek:sequence:for:set:iteration} is the simplest out of the presented methods to determine a maximal robust positive invariant set.
%
To illustrate the problem of treating the set iterations in a Hausdorff distance framework we first assume that we have no disturbances, i.e.
%
\begin{equation}
	\E_k = \Psi^{-k}\tilde\X
\end{equation}
%
with~$\Psi=A+BK$ and $\tilde\X=\X\cap K^{-1}\U$. 
%
Furthermore assume that the set~$\tilde\X=\conv\{v_i\}_{i\leq M}$, then clearly $\E_k = \conv\{\Psi^{-k}v_i\}_{i\leq M}$.
%
Studying convergence of 
%
\[
X_n = \bigcap_{k\leq n}\E_k
\]
%
becomes $d(X_n,X_{n+1})=d(X_n,X_n\cap \E_{n+1})$. 
%
And hence for $X_n$ to converge we require $d(X_n,\E_{n+1})\rightarrow0$, here we face a critical problem for the Hausdorff analysis of polytopes, the intersection of $\X=\conv\{v_i\}$ and $\Y=\conv\{w_i\}$ does not necessarily share vertices with either~$\X$ or~$\Y$ i.e. $\X\cap\Y=\conv\{r_i\}$ where $r_i\not\in\{v_i\}\cup\{w_i\}$ or $r_i\in\{v_i\}\cup\{w_i\}$, this is illustrated for simple polytopes in Figure~\ref{fig:intersection:of:polytopes}.
%
%
\begin{figure}\centering
\begin{tikzpicture}
\draw[red] (1,2) -- (1,-2) -- (-1,-2) -- (-1,2) -- cycle;
\draw[blue] (2,1) -- (2,-1) -- (-2,-1) -- (-2,1) -- cycle;
\draw (1,1) -- (1,-1) -- (-1,-1) -- (-1,1) -- cycle;
\draw[red] (0,1.5) node {$\X$};
\draw[blue] (1.5,0) node {$\Y$};
\end{tikzpicture}
\caption[Intersection of polytopes]{The intersection of two simple polytopes for which the vertices of the intersection are different to both vertex sets of~$\X$ and~$\Y$.}
\label{fig:intersection:of:polytopes}
\end{figure}
%
This then outlines why it becomes difficult to study the distance between~$X_n$ and $\E_{n+1}$ for the unperturbed case.
%
The case for non-trivial disturbances ($\W\neq\{0\}$) the sequence becomes even more complicated as
%
\[
\E_k = \Psi^{-k}\left(\tilde\X\ominus\bigoplus_{n=0}^{\max\{k-1,0\}}\Psi^n\W\right)
\]
%
involves more set operations, as discussed in Appendix~\ref{app:minkowski:pontryagin:identities} the Pontryagin difference of two sets in vertex representation requires the projection of a high dimensional polytope for which no relationship of the Hausdorff measure is known.
%
\\[1em]
%
In short: The same process that makes the maximal robust positive invariant set computation terminate in a finite number of steps (the intersection of exponentially expanding sets) makes the analysis of the transition difficult and conservative.
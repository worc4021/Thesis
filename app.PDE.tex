%!TEX root = main.tex

\chapter{The unconstrained min-max solution}

\begin{equation}
	V(x) = \min_u\max_w l(x,u) - \gamma(w) + V(f(x,u,w))
\end{equation}
%
introducing
%
\begin{equation}
	H(x,u) = \max_w -\gamma(w) + V(f(x,u,w))
\end{equation}
%
leads to
%
\begin{gather}
	V(x) = \min_u l(x,u) + H(x,u)\\
	H(x,u) = \max_w -\gamma(w) + V(f(x,u,w))
\end{gather}
%
and introducing the additional variable $z=f(x,u,w)$ yields
%
\begin{equation}
	H(x,u) = \begin{array}{rl}\max_{w,z} &-\gamma(w) + V(z)\\
	\text{s.t.}& z = f(x,u,w)\end{array}
\end{equation}
%
which then gives the Lagrange function
\begin{equation}
	L = -\gamma(w) + V(z) -\lambda^T(z-f(x,u,w))
\end{equation}
%
So that the first order conditions for the maximisation are given by
%
\begin{gather}
-\nabla_w \gamma + f_w^T\lambda = 0\\
\nabla_z V -\lambda = 0\\
z-f(x,u,w) = 0
\end{gather}
%
and for the minimisation we have
%
\begin{equation}
	\nabla_u l + \nabla_u H = 0
\end{equation}
%
The implicit function theorem implies that $w^\ast = w(x,u)$ and $z^\ast = z(x,u)$ and hence
%
\begin{equation}
	H(x,u) = -\gamma(w(x,u)) + V(z(x,u))
\end{equation}
%
so that 
%
\begin{equation}\left.\begin{split}
	\nabla_u H &= - w_u^T\nabla_w\gamma + z_u \nabla_z V\\
	z_u &= f_u + f_w w_u
\end{split}\right\} \Rightarrow \nabla_u l -w_u^T \nabla_w\gamma + (f_u+f_w w_u)^T\nabla_z V = 0.
\end{equation}
%
For convexity of the minimisation we require $l_{uu}+H_{uu}>0$, this is then given by
%
\begin{equation}\begin{split}
	\frac{\partial H}{\partial u^2} = \frac{\partial}{\partial u} \left(- w_u^T\nabla_w\gamma + z_u \nabla_z V\right)
	= -w_u^T \nabla_{ww}\gamma w_u - \left(\begin{array}{ccc}
	w_{u{u_1}}^T\nabla_w\gamma & \dots & w_{u{u_{N_u}}}^T\nabla_w\gamma
	\end{array}\right)+\\ z_u^T\nabla_{zz} V z_u + \left(\begin{array}{ccc}
	z_{uu_1}^T\nabla_z V & \dots & z_{uu_{N_u}}^T\nabla_z V
	\end{array}\right)\\
	=w_u^T\left(f_w^T\nabla_{zz}Vf_w-\nabla_{ww}\gamma\right)w_u + f_u^T\nabla_{zz}Vf_w w_u + w_u^Tf_w^T\nabla_{zz}V f_u
	+ f_u^T\nabla_{zz}Vf_u\\
	+\underbrace{\left(\begin{array}{ccc}z_{uu_i}^T\nabla_z V &\dots & z_{uu_{N_u}}^T \nabla_z V
	\end{array}\right)-\left(\begin{array}{ccc}
	w_{uu_1}^T\nabla_w \gamma & \dots & w_{uu_{N_u}}^T\nabla_w \gamma
	\end{array}\right)}_{\mathscr A}.
\end{split}\end{equation}
%
To derive a condition independent of $w$ and $z$ we derive an identity for $w_u$:
%
\begin{equation}\begin{split}
	\frac{\partial}{\partial u} (f_w^T\lambda-\nabla_w\gamma) &= 0
\end{split}\end{equation}
%
using the optimal value $\lambda^\ast=\nabla_z V(z(x,u))$ and hence $\lambda_u = \nabla_{zz}V z_u$.
%
So that we have
%
\begin{equation}
	\left(\begin{array}{ccc} f_{wu_1}^T \nabla_z V & \dots & f_{wu_{N_u}}^T\nabla_z V \end{array}\right) + f_w^T\nabla_{zz}V z_u
	-\nabla_{ww}\gamma w_u=0
\end{equation}
%
and with $z_u = f_u + f_w w_u$ we get
%
\begin{equation}
	\underbrace{\left(\begin{array}{ccc} f_{wu_1}^T \nabla_z V & \dots & f_{wu_{N_u}}^T\nabla_z V \end{array}\right) + f_w^T\nabla_{zz}V f_u}_L =
	\underbrace{(\nabla_{ww}\gamma-f_w^T\nabla_{zz}V f_w)}_R w_u.
\end{equation}
%
% We can now write $w_u = R^{-1}L$ where neither $R$ nor $L$ depends on $w$ or $z$.
% %
In order to derive a condition to guarantee $l_{uu}+H_{uu}>0$ we need an identity for $z_{uu_i}$, this
can be found straight forward using
%
\begin{equation}
	z_{uu_i} = f_{uu_i} + \left(\begin{array}{ccc}
	f_{u_1w}w_{u_i} & \dots & f_{u_{N_u}w} w_{u_i}
	\end{array}\right) + f_{wu_i}w_u + \left(\begin{array}{ccc}
	f_{w_1w} w_{u_i} & \dots & f_{w_{N_w}w}w_{u_i}
	\end{array}\right)w_u + f_w w_{uu_i}.
\end{equation}
%
and therefore
%
\begin{equation}
	\nabla V^T z_{uu_i} = \nabla V^T f_{uu_i} + \left(\begin{array}{ccc}\nabla V^T f_{u_1w}w_{u_i}&\dots 
	&\nabla V^T f_{u_{N_u}w}w_{u_i}\end{array}\right)
\end{equation}
% %
% We can easily derive an identity for $w_{u_i}$, namely
% %
% \begin{equation}
% 	f_{wu_i}^T\nabla_z V + f_w^T \nabla_{zz}V f_{u_i} = R w_{u_i}.
% \end{equation}
% %
% We will abbreviate the left hand side by $L_i$, i.e. $w_{u_i} = R^{-1}L_i$.
% %
% So that $z_{uu_i}$ can be rewritten as
% %
% \begin{multline}
% 	z_{uu_i} = f_{uu_i} + \left(\begin{array}{ccc}
% 	f_{u_1w}R^{-1}L_i & \dots & f_{u_{N_u}w} R^{-1}L_i
% 	\end{array}\right) + f_{wu_i}R^{-1}L\\ + \left(\begin{array}{ccc}
% 	f_{w_1w} R^{-1}L_i & \dots & f_{w_{N_w}w}R^{-1}L_i
% 	\end{array}\right)R^{-1}L + f_w w_{uu_i}
% \end{multline}



% Notice that 
% %
% \begin{equation}
% 	z_u = \left(\begin{array}{cccc}
% 	z_{1,u_1} & z_{1,u_2} & \dots & z_{1,u_{N_u}}\\
% 	\vdots & & & \vdots \\
% 	z_{n,u_1} & z_{n,u_2} & \dots & z_{n,u_{N_u}}
% 	\end{array}\right)
% \end{equation}
% %
% is a $n\times N_u$ Jacobian matrix and 
% %
% \begin{equation}
% 	\partial_{u_1}z_u = 
% 	\left(\begin{array}{cccc}
% 	z_{1,u_1^2} & z_{1,u_2u_1} & \dots & z_{1,u_{N_u}u_1}\\
% 	\vdots & & & \vdots \\
% 	z_{n,u_1^2} & z_{n,u_2u_1} & \dots & z_{n,u_{N_u}u_1}
% 	\end{array}\right)
% \end{equation}
% %
% is again a $n\times N_u$ matrix so that 
% %
% \begin{equation}
% 	V_z\partial_{u_1} z_u = \left(\begin{array}{cccc} 
% 	V_z z_{u_1^2} & V_z z_{u_2u_1} & \dots& V_z z_{u_{N_u}u_1}
% 	\end{array}\right)
% \end{equation}
% %
% is a $1\times N_u$ and so 
% %
% \begin{equation}
% 	\left(\begin{array}{c}
% 	V_z \partial_{u_1} z_u \\ \vdots \\ V_z \partial_{u_{N_u}} z_u
% 	\end{array}\right) = 
% 	\left(\begin{array}{cccc}
% 	V_z z_{u_1^2} & V_z z_{u_2u_1} & \dots &V_z z_{u_{N_u}u_1} \\
% 	V_z z_{u_1u_2} & V_z z_{u_2^2} & \dots& V_z z_{u_{N_u}u_2} \\
% 	\vdots & & & \vdots \\
% 	V_z z_{u_1u_{N_u}} & V_z z_{u_2u_{N_u}} & \dots& V_z z_{u_{N_u}^2}
% 	\end{array}\right).
% \end{equation}
% %
% Since $V_z z_{u_iu_j}$ is a scalar we have
% %
% \begin{multline}
% 	\left(\begin{array}{c}
% 	V_z \partial_{u_1} z_u \\ \vdots \\ V_z \partial_{u_{N_u}} z_u
% 	\end{array}\right)^T = 
% 	\left(\begin{array}{cccc}
% 	V_z z_{u_1^2} & V_z z_{u_2u_1} & \dots &V_z z_{u_{N_u}u_1} \\
% 	V_z z_{u_1u_2} & V_z z_{u_2^2} & \dots& V_z z_{u_{N_u}u_2} \\
% 	\vdots & & & \vdots \\
% 	V_z z_{u_1u_{N_u}} & V_z z_{u_2u_{N_u}} & \dots& V_z z_{u_{N_u}^2}
% 	\end{array}\right)^T
% 	= \\
% 	\left(\begin{array}{cccc}
% 	V_z z_{u_1^2} & V_z z_{u_1u_2} & \dots &V_z z_{u_1u_{N_u}} \\
% 	V_z z_{u_2u_1} & V_z z_{u_2^2} & \dots& V_z z_{u_2u_{N_u}} \\
% 	\vdots & & & \vdots \\
% 	V_z z_{u_{N_u}u_1} & V_z z_{u_{N_u}u_2} & \dots& V_z z_{u_{N_u}^2}
% 	\end{array}\right)  = 
% 	\left(\begin{array}{cccc}
% 	V_z z_{u_1^2} & V_z z_{u_2u_1} & \dots &V_z z_{u_{N_u}u_1} \\
% 	V_z z_{u_1u_2} & V_z z_{u_2^2} & \dots& V_z z_{u_{N_u}u_2} \\
% 	\vdots & & & \vdots \\
% 	V_z z_{u_1u_{N_u}} & V_z z_{u_2u_{N_u}} & \dots& V_z z_{u_{N_u}^2}
% 	\end{array}\right) \\
% 	= \left(\begin{array}{c}
% 	V_z \partial_{u_1} z_u \\ \vdots \\ V_z \partial_{u_{N_u}} z_u
% 	\end{array}\right).
% \end{multline}
% %
% And so all terms of $H_{uu}$ are symmetric and none can be discarded in advance.

%
% or
% %
% \begin{multline}
% 	\frac{\partial H}{\partial u^2} = -w_u^T \gamma_{ww} w_u + (f_u + f_w w_u)^T V_{zz}(f_u + f_w w_u) + (I_{N_u}\otimes V_z)
% 	\left(\begin{array}{c}
% 	\partial_{u_1}z_u \\ \vdots \\ \partial_{u_{N_u}} z_u
% 	\end{array}\right)\\
% 	-(I_{N_u}\otimes\gamma_w)\left(\begin{array}{c}
% 	\partial_{u_1} w_u\\
% 	\vdots \\
% 	\partial_{u_{N_u}} w_u
% 	\end{array}\right)\\
% 	=\left(\begin{array}{c}I_{N_u} \\ w_u \end{array}\right)^T
% 	\left(\begin{array}{cc}
% 	f_u^TV_{zz}f_u & f_u^T V_{zz} \\
% 	V_{zz}f_u & (f_w^T V_{zz}f_w - \gamma_w)
% 	\end{array}\right)
% 	\left(\begin{array}{c}I_{N_u} \\ w_u \end{array}\right)
% 	+ (I_{N_u}\otimes V_z)
% 	\left(\begin{array}{c}
% 	\partial_{u_1}z_u \\ \vdots \\ \partial_{u_{N_u}} z_u
% 	\end{array}\right)\\
% 	-(I_{N_u}\otimes\gamma_w)\left(\begin{array}{c}
% 	\partial_{u_1} w_u\\
% 	\vdots \\
% 	\partial_{u_{N_u}} w_u
% 	\end{array}\right)
% \end{multline}



% %
% \begin{equation}
% 	\nabla_u \bigl( l(x,u) + H(x,u) \bigr) = 0 \wedge \nabla_w \bigl(-\gamma(w) + V(f(x,u,w))\bigr) = 0
% \end{equation}
% %
% \begin{gather}
% 	l_u(x,u) + H_u(x,u) = 0 \\
% 	-\gamma_w(w) + V_x(f(x,u,w))f_w(x,u,w) = 0
% \end{gather}
% %
% Which implies that $w^\ast = w(x,u)$ and~$u^\ast = u(x)$, so that 
% %
% \begin{gather}
% 	V(x) = l(x,u(x)) + H(x,u(x))\\
% 	H(x,u) = -\gamma(w(x,u)) + V(f(x,u,w(x,u)))
% \end{gather}
% %
% And hence
% %
% \begin{multline}
% 	H_u(x,u) = -\gamma_w(w(x,u))w_u(x,u) + V_x(f(x,u,w(x,u)))f_u(x,u,w(x,u))\\
% 	 + V_x(f(x,u,w(x,u)))f_w(x,u,w(x,u))w_u(x,u),
% \end{multline}
% %
% \begin{equation}
% 	V_x(x) = l_x(x,u(x)) + l_u(x,u(x)) u_x(x) + H_x(x,u(x)) + H_u(x,u(x))u_x(x)
% \end{equation}
% %
% \begin{multline}
% 	H_x(x,u) = -\gamma_w(w(x,u))w_x(x,u) + V_x(f(x,u,w(x,u))f_u(x,u,w(x,u)) +\\
% 	 V_x(f(x,u,w(x,u)))f_w(x,u,w(x,u))w_x(x,u)
% \end{multline}
% %
% This leads to
% %
% \begin{equation}\begin{split}
% 	V_x &= l_x + l_u u_x + -\gamma_w w_x + V_x f_u + V_x f_w w_x - \gamma_w w_u + V_x f_u + V_x f_w w_u\\
% 	V_x(I - f_u - f_w (w_x + w_u)) &= l_x + l_u u_x - \gamma_w (w_x + w_u)\\
% 	V_x &= (l_x + l_u u_x - \gamma_w (w_x + w_u)) (I - f_u - f_w (w_x + w_u))^{-1}.
% \end{split}\end{equation}
% %
% So that $w(x,u)$ satisfies the partial differential equation
% %
% \begin{equation}
% 	\gamma_w = (l_x + l_u u_x - \gamma_w (w_x + w_u)) (I - f_u - f_w (w_x + w_u))^{-1}f_w
% \end{equation}
% %
% the optimal control strategy $u(x)$ satisfies
% %
% \begin{equation}
% 	l_u - \gamma_w w_u + (l_x + l_u u_x - \gamma_w (w_x + w_u)) (I - f_u - f_w (w_x + w_u))^{-1}(f_u + f_w w_u) = 0
% \end{equation}

% We require $V_{xx}\geq0$, hence
% %
% \begin{equation}
% 	V_{xx} = \left(\begin{array}{c} I \\ u_x \end{array}\right)^T
% 	\left(\left(\begin{array}{cc}
% 	l_{xx} & l_{xu}\\
% 	l_{ux} & l_{uu}
% 	\end{array}\right)
% 	+
% 	\left(\begin{array}{cc}
% 	H_{xx} & H_{xu} \\
% 	H_{ux} & H_{uu}
% 	\end{array}\right)\right)
% 	\left(\begin{array}{c} I \\ u_x \end{array}\right)
% 	+
% 	(l_u + H_u)u_{xx}
% \end{equation}

% \begin{multline}
% H_{xu} = -\gamma_{w} w_{xu} - w_u^T \gamma_{ww} w_u + V_x( f_{xu}+f_{wu} w_x + w_u^T (f_{xw} + f_{ww} w_u) )
% \end{multline}
